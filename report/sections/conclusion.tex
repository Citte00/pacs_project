This report has provided a comprehensive study of adaptive mesh refinement strategies within the context of the Discontinuous Galerkin (DG) method, focusing on both \textit{h-adaptivity} and \textit{p-adaptivity}.

The evaluation of \textit{h-adaptive} refinement strategies demonstrated their effectiveness in managing complex geometries and solutions with local singularities. This approach effectively reduces local discretization errors by refining the mesh where needed.

In the domain of \textit{p-adaptivity}, the analysis of the decay rates of Legendre coefficients proved to be a reliable method for assessing solution smoothness and guiding polynomial order adjustments.

A comparison of \textit{h-adaptive} and \textit{hp-adaptive} refinement strategies highlighted their respective advantages. While \textit{h-adaptivity} effectively manages local errors through mesh refinement, \textit{hp-adaptivity} offers additional benefits by adjusting polynomial orders alongside mesh resolution.

Overall, the results of this project underscore the practical benefits of adaptive refinement strategies in DG methods. By utilizing a posteriori error estimators, \textit{h-adaptivity} and \textit{hp-adaptivity} can be effectively implemented to achieve accurate and efficient numerical solutions.